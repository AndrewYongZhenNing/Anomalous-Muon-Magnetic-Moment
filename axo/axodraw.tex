\documentstyle[11pt,a4,axodraw]{article}

\def\KeyWord#1{$\backslash$\IfColor{$\!\!$\textRed{#1}\textBlack}{#1}$\!\!$}

\begin{document}

\begin{center}
{\Huge \IfColor{\textRed{A}\textOrange{X}\textYellow{O}\textGreen
{D}\textBlue{R}\textRoyalPurple{A}\textViolet{W}\textBlack$\!\!\!$}{AXODRAW}} \\
\vspace{3cm}
{\LARGE  J.A.M.Vermaseren} \\
\vspace{4mm}
NIKHEF-H \\ P.O. Box 41882 \\ 1009 DB Amsterdam \\ \vspace{5mm}
\end{center}

\vspace{5cm}

\begin{abstract}
Axodraw is a set of drawing primitives for use in \LaTeX. These can be used 
for the drawing of Feynman diagrams, flow charts and simple graphics. 
Because it uses postscript for its drawing commands it works only in 
combination with the dvips of Radical Eye Software which is presently the 
most popular dvips program. More will be added in the future. It allows 
whole articles including their pictures to be contained in a single file, 
thereby making it easier to exchange the article file by e-mail. The 
current version\footnote{An earlier version of Axodraw was published 
in Comp. Phys. Comm. 83 (1994) 45.} supports color according to the scheme 
implemented in the file colordvi.sty which comes with most \TeX{} 
distributions.
\end{abstract}

\newpage

\section{Using Axodraw}

The file axodraw.sty is a style file for \LaTeX{}. It should be included 
in the documentstyle statement at the beginning of the 
document. An example would be: \IfColor{\textBlue}{}
\begin{verbatim}
    \documentstyle[a4,11pt,axodraw]{article}
\end{verbatim}
\IfColor{\textBlack}{}
Because axodraw.sty reads also the epsf.sty file that comes with many 
implementations of \TeX{} and in particular those that rely on the dvips 
program by Radical Eye Software for the printing, this file should be 
present in the system. If this file is not available one should obtain it 
from another system. The file colordvi.sty is also read, but if it is not 
present there will be no error. The user should just not use color in that 
case. The author feels in no way responsible for the problems that may 
occur when a different dvi-to-postscript program is used.

The drawing is actually done in postscript. Because the above mentioned 
dvi-to-postscript converter allows the inclusion of postscript code the 
graphics primitives have been included in the file axodraw.sty in terms 
of postscript. If another postscript converter is used, one may have to 
adapt the syntax of the inclusion of this code to the local system.

The commands of Axodraw should be executed inside either the picture or the 
figure environment. Inside this environment it is possible to place objects 
at arbitrary positions and put text between them. In principle one could 
try to draw objects with the facilities of \LaTeX{} itself, but it turns 
out that the commands in the picture environment are not very powerful. 
Axodraw gives good extensions of them. An example would be
\IfColor{\textBlue}{}
\begin{verbatim}
    \begin{center} \begin{picture}(300,100)(0,0)
    \SetColor{Red}
    \GlueArc(150,50)(40,0,180){5}{8}
    \SetColor{Green}
    \GlueArc(150,50)(40,180,360){5}{8}
    \SetColor{Blue}
    \Gluon(50,50)(110,50){5}{4}  \Vertex(110,50){2} 
    \Gluon(190,50)(250,50){5}{4} \Vertex(190,50){2}
    \end{picture}  \\ {\sl A gluon loop diagram} \end{center}
\end{verbatim}
\IfColor{\textBlack}{}
This code would result in:
\begin{center} \begin{picture}(300,100)(0,0)
\SetColor{Red}
\GlueArc(150,50)(40,0,180){5}{8}
\SetColor{Green}
\GlueArc(150,50)(40,180,360){5}{8}
\SetColor{Blue}
\Gluon(50,50)(110,50){5}{4}  \Vertex(110,50){2}
\Gluon(190,50)(250,50){5}{4} \Vertex(190,50){2}
\end{picture}  \\ {\sl A gluon loop diagram} \end{center}
The syntax and the meaning of these command are explained in the next 
section. One should note that all coordinates are presented in units of 
1 point. There are 72 points in an inch. It is possible to use scale 
transformations if these units are not convenient.

Currently the primitives are mainly useful for the drawing of Feynman 
diagrams and the drawing of flowcharts. This means that the commands were 
designed to draw a number of these graphs. Of course many more things can 
be drawn with them, like scatter plots, histograms etc.

The current manual uses only those color commands that are safe on systems 
that do not have the required colordvi.sty file. It should however be clear 
from the examples how to use the other features. To allow the creation of 
complicated color commands that will work also in the absence of the 
colordvi.sty file there is a macro IfColor which is described below with 
the other commands.

\section{The commands}

The commands that are currently available in Axodraw are (in alphabetic 
order):
\begin{itemize}
\item \KeyWord{ArrowArc}(x,y)(r,$\phi_1$,$\phi_2$) \hfill \\
    Draws an arc segment centered around (x,y). The radius is r. The 
    arc-segment runs counterclockwise from $\phi_1$ to $\phi_2$. All 
    angles are given in degrees. In the middle of the segment there will 
    be an arrow.
\item \KeyWord{ArrowArcn}(x,y)(radius,$\phi_1$,$\phi_2$) \hfill \\
    Draws an arc segment centered around (x,y). The radius is r. The 
    arc-segment runs clockwise from $\phi_1$ to $\phi_2$. All 
    angles are given in degrees. In the middle of the segment there will 
    be an arrow.
\item \KeyWord{ArrowLine}($x_1$,$y_1$)($x_2$,$y_2$) \hfill \\
    Draws a line from ($x_1$,$y_1$) to ($x_2$,$y_2$). There will be an 
    arrow in the middle of the line.
\item \KeyWord{BBox}($x_1$,$y_1$)($x_2$,$y_2$) \hfill \\
    Draws a box of which the contents are blanked out. This means that 
    anything that was present at the position of the box will be 
    overwritten. The lower left corner of the box is at ($x_1$,$y_1$) and 
    ($x_2$,$y_2$) is the upper right corner of the box.
\item \KeyWord{BBoxc}(x,y)(width,height) \hfill \\
    Draws a box of which the contents are blanked out. This means that 
    anything that was present at the position of the box will be 
    overwritten. The center of the box is at (x,y). Width and height refer 
    to the full width and the full height of the box.
\item \KeyWord{BCirc}(x,y)\{r\} \hfill \\
    Draws a circle of which the contents are blanked out. This means that 
    anything that was present at the position of the circle will be 
    overwritten. The center of the circle is at (x,y). r is its radius.
\item \KeyWord{Boxc}(x,y)(width,height) \hfill \\
    Draws a box. The center of the box is at (x,y). Width and height refer 
    to the full width and the full height of the box.
\item \KeyWord{BText}(x,y)\{text\} \hfill \\
    Draws a box with one line of centered postscript text in it. The box is 
    just big enough to fit around the text. The coordinates refer to the 
    center of the box. The box is like a BBox in that it blanks out 
    whatever was at the position of the box.
\item \KeyWord{B2Text}(x,y)\{text1\}\{text2\} \hfill \\
    Draws a box with two lines of centered postscript text in it. The box is 
    just big enough to fit around the text. The coordinates refer to the 
    center of the box. The box is like a BBox in that it blanks out 
    whatever was at the position of the box.
\item \KeyWord{CArc}(x,y)(radius,$\phi_1$,$\phi_2$) \hfill \\
    Draws an arc segment centered around (x,y). The radius is r. The 
    arc-segment runs counterclockwise from $\phi_1$ to $\phi_2$. All 
    angles are given in degrees.
\item \KeyWord{CBox}($x_1$,$y_1$)($x_2$,$y_2$)\{color1\}\{color2\} \hfill \\
    Draws a box. The lower left corner of the box is at ($x_1$,$y_1$) and 
    ($x_2$,$y_2$) is the upper right corner of the box. The contents of 
    the box are lost. The color of the box will be color1 and the color of 
    the background inside the box will be color2.
\item \KeyWord{CBoxc}(x,y)(width,height)\{color1\}\{color2\} \hfill \\
    Draws a box of which the contents are blanked out. This means that 
    anything that was present at the position of the box will be 
    overwritten. The center of the box is at (x,y). Width and height refer 
    to the full width and the full height of the box. The color of the box 
    will be color1 and the color of the background inside the box will be 
    color2.
\item \KeyWord{CCirc}(x,y)\{radius\}\{color1\}\{color2\} \hfill \\
    Draws a circle around (x,y) with radius r. The contents of 
    the circle are lost. The color of the box will be color1 and the color 
    of the background inside the box will be color2.
\item \KeyWord{COval}(x,y)(h,w)($\phi$)\{color1\}\{color2\} \hfill \\
    Draws an oval with an internal color indicated by color2. The oval 
    itself has the color color1. The center of the oval 
    is given by (x,y). Its height is h, and the width is w. In addition 
    the oval can be rotated counterclockwise over $\phi$ degrees. The 
    oval overwrites anything that used to be in its position.
\item \KeyWord{CText}(x,y)\{color1\}\{color2\}\{text\} \hfill \\
    Draws a box with one line of centered postscript text in it. The 
    box is just big enough to fit around the text. The coordinates refer to 
    the center of the box. The box is like a CBox in that it blanks out 
    whatever was at the position of the box. The color of the box and the 
    text inside is color1 and the background inside has the color color2.
\item \KeyWord{C2Text}(x,y)\{color1\}\{color2\}\{text1\}\{text2\} \hfill \\
    Draws a box with two lines of centered postscript text in it. 
    The box is just big enough to fit around the text. The coordinates 
    refer to the center of the box. The box is like a CBox in that it blanks 
    out whatever was at the position of the box. The color of the box and the 
    text inside is color1 and the background inside has the color color2.
\item \KeyWord{Curve}\{$(x_1,y_1)(x_2,y_2)\cdots(x_n,y_n)$\} \hfill \\
    Draws a curve through the given points. The x-values are supposed to 
    be in ascending order. The curve is a combination of quadratic and 
    third order segments and is continuous in its first and second derivatives.
\item \KeyWord{DashArrowArc}(x,y)(r,$\phi_1$,$\phi_2$)\{dashsize\} \hfill \\
    Draws a dashed arc segment centered around (x,y). The radius is r. The 
    arc-segment runs counterclockwise from $\phi_1$ to $\phi_2$. All 
    angles are given in degrees. In the middle of the segment there will 
    be an arrow. The size of the dashes is approximately equal to 
    `dashsize'.
\item \KeyWord{DashArrowArcn}(x,y)(radius,$\phi_1$,$\phi_2$)\{dashsize\} \hfill \\
    Draws a dashed arc segment centered around (x,y). The radius is r. The 
    arc-segment runs clockwise from $\phi_1$ to $\phi_2$. All 
    angles are given in degrees. In the middle of the segment there will 
    be an arrow. The size of the dashes is approximately equal to 
    `dashsize'.
\item \KeyWord{DashArrowLine}($x_1$,$y_1$)($x_2$,$y_2$)\{dashsize\} \hfill \\
    Draws a line from ($x_1$,$y_1$) to ($x_2$,$y_2$) with a dashed 
    pattern. The size of the black parts of the pattern is given by 
    `dashsize'. The alternating pieces have equal length. The size of 
    the pattern is adjusted so that both the begin and the end are 
    black. Halfway the line there is an arrow.
\item \KeyWord{DashCArc}(x,y)(radius,$\phi_1$,$\phi_2$)\{dashsize\} \hfill \\
    Draws a dashed arc segment centered around (x,y). The radius is r. The 
    arc-segment runs counterclockwise from $\phi_1$ to $\phi_2$. All 
    angles are given in degrees. The size of the dashes is determined by 
    `dashsize'. This size is adjusted somewhat to make the result look 
    nice.
\item \KeyWord{DashCurve}\{$(x_1,y_1)(x_2,y_2)\cdots(x_n,y_n)$\}\{
    dashsize\} \hfill \\
    Draws a dashed curve through the given points. The x-values are 
    supposed to be in ascending order. The curve is a combination of 
    quadratic and third order segments. The size of the black parts and 
    the white parts will be approximately `dashsize' each. Some 
    adjustment takes place to make the pattern come out right at the 
    endpoints.
\item \KeyWord{DashLine}($x_1$,$y_1$)($x_2$,$y_2$)\{dashsize\} \hfill \\
    Draws a line from ($x_1$,$y_1$) to ($x_2$,$y_2$) with a dashed 
    pattern. The size of the black parts of the pattern is given by 
    `dashsize'. The alternating pieces have equal length. The size of 
    the pattern is adjusted so that both the begin and the end are 
    black.
\item \KeyWord{EBox}($x_1$,$y_1$)($x_2$,$y_2$) \hfill \\
    Draws a box. The lower left corner of the box is at ($x_1$,$y_1$) and 
    ($x_2$,$y_2$) is the upper right corner of the box.
\item \KeyWord{IfColor}\{arg1\}\{arg2\} \hfill \\
    If the file colordvi.sty is present the first argument will be 
    executed. If this file is not present the second argument will be 
    executed. For examples, see some of the figures. This command can also 
    be used in the regular text of the \LaTeX{} file.
\item \KeyWord{GBox}($x_1$,$y_1$)($x_2$,$y_2$)\{grayscale\} \hfill \\
    Draws a box. The lower left corner of the box is at ($x_1$,$y_1$) and 
    ($x_2$,$y_2$) is the upper right corner of the box. The contents of 
    the box are lost. They are overwritten with a color gray that is 
    indicated by the parameter `grayscale'. This parameter can have 
    values ranging from 0 (black) to 1 (white).
\item \KeyWord{GBoxc}(x,y)(width,height)\{grayscale\} \hfill \\
    Draws a box. The center of the box is at (x,y). Width and height refer 
    to the full width and the full height of the box. The contents of 
    the box are lost. They are overwritten with a color gray that is 
    indicated by the parameter `grayscale'. This parameter can have 
    values ranging from 0 (black) to 1 (white).
\item \KeyWord{GCirc}(x,y)\{radius\}\{grayscale\} \hfill \\
    Draws a circle around (x,y) with radius r. The contents of 
    the circle are lost. They are overwritten with a color gray that is 
    indicated by the parameter `grayscale'. This parameter can have 
    values ranging from 0 (black) to 1 (white).
\item \KeyWord{GlueArc}(x,y)(r,$\phi_1$,$\phi_2$)\{amplitude\}\{windings\} \hfill \\
    Draws a gluon on an arc-segment. The center of the arc is (x,y) and r 
    is its radius. The arc segment runs counterclockwise from $\phi_1$ 
    to $\phi_2$. The width of the gluon is twice `amplitude', and the 
    number of windings is given by the last parameter. Note that whether 
    the curls are inside or outside can be influenced with the sign of 
    the amplitude. When it is positive the curls are on the inside.
\item \KeyWord{Gluon}($x_1$,$y_1$)($x_2$,$y_2$)\{amplitude\}\{windings\} \hfill \\
    Draws a gluon from ($x_1$,$y_1$) to ($x_2$,$y_2$). The width of the 
    gluon will be twice the value of `amplitude'. The number of windings 
    is given by the last parameter. If this parameter is not an integer 
    it will be rounded to an integer value. The side at which the 
    windings lie is determined by the order of the two coordinates. Also 
    a negative amplitude can change this side.
\item \KeyWord{GOval}(x,y)(h,w)($\phi$)\{grayscale\} Draws an oval with
    an internal color indicated by grayscale. This parameter can have 
    values ranging from 0 (black) to 1 (white). The center of the oval 
    is given by (x,y). Its height is h, and the width is w. In addition 
    the oval can be rotated counterclockwise over $\phi$ degrees. The 
    oval overwrites anything that used to be in its position.
\item \KeyWord{GText}(x,y)\{grayscale\}\{text\} \hfill \\
    Draws a gray box with one line of centered postscript text in it. The 
    box is just big enough to fit around the text. The coordinates refer to 
    the center of the box. The box is like a BBox in that it blanks out 
    whatever was at the position of the box.
\item \KeyWord{G2Text}(x,y)\{grayscale\}\{text1\}\{text2\} \hfill \\
    Draws a gray box with two lines of centered postscript text in it. 
    The box is just big enough to fit around the text. The coordinates 
    refer to the center of the box. The box is like a BBox in that it 
    blanks out whatever was at the position of the box.
\item \KeyWord{LinAxis}($x_1$,$y_1$)($x_2$,$y_2$)($N_D$,$d$,hashsize
    ,offset,width) \hfill \\
    This draws a line to be used as an axis in a graph. Along the axis 
    are hash marks. Going from the first coordinate to the second, the 
    hash marks are on the left side if `hashsize', which is the size of the 
    hash marks, is positive and on the right side if it is negative. 
    $N_D$ is the number of `decades', indicated by fat hash marks, and 
    $d$ is the number of subdivisions inside each decade. The offset 
    parameter tells to which subdivision the first coordinate 
    corresponds. When it is zero, this coordinate corresponds to a fat 
    mark of a decade. Because axes have their own width, this is 
    indicated with the last parameter.
\item \KeyWord{Line}($x_1$,$y_1$)($x_2$,$y_2$) \hfill \\
    Draws a line from ($x_1$,$y_1$) to ($x_2$,$y_2$).
\item \KeyWord{LogAxis}($x_1$,$y_1$)($x_2$,$y_2$)($N_L$,hashsize
    ,offset,width) \hfill \\
    This draws a line to be used as an axis in a graph. Along the axis 
    are hash marks. Going from the first coordinate to the second, the 
    hash marks are on the left side if `hashsize', which is the size of the 
    hash marks, is positive and on the right side if it is negative. 
    $N_L$ is the number of orders of magnitude, indicated by fat 
    hash marks. The offset parameter tells to which integer subdivision 
    the first coordinate corresponds. When it is zero, this coordinate 
    corresponds to a fat mark, which is identical to when the value 
    would have been 1. Because axes have their own width, this is 
    indicated with the last parameter.
\item \KeyWord{LongArrow}($x_1$,$y_1$)($x_2$,$y_2$) \hfill \\
    Draws a line from ($x_1$,$y_1$) to ($x_2$,$y_2$). There will be an 
    arrow at the end of the line.
\item \KeyWord{LongArrowArc}(x,y)(r,$\phi_1$,$\phi_2$) \hfill \\
    Draws an arc segment centered around (x,y). The radius is r. The 
    arc-segment runs counterclockwise from $\phi_1$ to $\phi_2$. All 
    angles are given in degrees. At the end of the segment there will 
    be an arrow.
\item \KeyWord{LongArrowArcn}(x,y)(radius,$\phi_1$,$\phi_2$) \hfill \\
    Draws an arc segment centered around (x,y). The radius is r. The 
    arc-segment runs clockwise from $\phi_1$ to $\phi_2$. All 
    angles are given in degrees. At the end of the segment there will 
    be an arrow.
\item \KeyWord{Oval}(x,y)(h,w)($\phi$) Draws an oval.
    The center of the oval 
    is given by (x,y). Its height is h, and the width is w. In addition 
    the oval can be rotated counterclockwise over $\phi$ degrees. The 
    oval does not overwrite its contents.
\item \KeyWord{Photon}($x_1$,$y_1$)($x_2$,$y_2$)\{amplitude\}\{wiggles\} \hfill \\
    Draws a photon from ($x_1$,$y_1$) to ($x_2$,$y_2$). The width of the 
    photon will be twice the value of `amplitude'. The number of wiggles 
    is given by the last parameter. If twice this parameter is not an integer 
    it will be rounded to an integer value. Whether the first wiggle 
    starts up or down can be influenced with the sign of the amplitude.
\item \KeyWord{PhotonArc}(x,y)(r,$\phi_1$,$\phi_2$)\{amplitude\}\{wiggles\} \hfill \\
    Draws a photon on an arc-segment. The center of the arc is (x,y) and r 
    is its radius. The arc segment runs counterclockwise from $\phi_1$ 
    to $\phi_2$. The width of the photon is twice `amplitude', and the 
    number of wiggles is given by the last parameter. Note that 
    the sign of the amplitude influences whether the photon starts going 
    outside (positive) or starts going inside (negative). If one likes 
    the photon to reach both endpoints from the outside the number of 
    wiggles should be an integer plus 0.5.
\item \KeyWord{PText}(x,y)($\phi$)$[$mode$]$\{text\} \hfill \\
    Places a postscript text. The focal point is (x,y). The text is the last 
    parameter. The mode parameter tells how the text should be 
    positioned with respect to the focal point. If this parameter is 
    omitted the center of the text will correspond to the focal point. 
    Other options are: l for having the left side correspond to the 
    focal point, r for having the right side correspond to it, t for 
    having the top at the focal point and b for the bottom. One may 
    combine two letters as in $[$bl$]$, as long as it makes sense. The 
    parameter $\phi$ is a rotation angle. The text is written in the 
    current postscript font. This font can be set with the SetPFont 
    command.
\item \KeyWord{rText}(x,y)$[$mode$][$rotation$]$\{text\} \hfill \\
    Places a rotated text. The focal point is (x,y). The text is the last 
    parameter. If the rotation parameter is the character l the text 
    will be rotated left by 90 degrees, if it is an r it will be rotated 
    to the right by 90 degrees and when it is the character u the text 
    will be rotated by 180 degrees. When there is no character there is 
    no rotation and the command is identical to the Text command. 
    The mode parameter tells how the resulting box should be 
    positioned with respect to the focal point. If this parameter is 
    omitted the center of the box will correspond to the focal point. 
    Other options are: l for having the left side correspond to the 
    focal point, r for having the right side correspond to it, t for 
    having the top at the focal point and b for the bottom. One may 
    combine two letters as in $[$bl$]$, as long as it makes sense.
\item \KeyWord{SetColor}\{NameOfColor\} \hfill \\
    Sets the color for the next commands. This command onlt affects the 
    current picture. In addition it does not affect the text commands that 
    write in \TeX{} mode. Also the commands that draw gray boxes are not 
    affected. For influencing the color of the \TeX{} or \LaTeX{} output 
    one can use the commands mentioned in the colordvi.sty file.
\item \KeyWord{SetPFont}\{fontname\}\{fontsize\} \hfill \\
    Sets the postscript font to a given type and scale.
\item \KeyWord{SetScale}\{scalevalue\} \hfill \\
    Changes the scale of all graphics operations. Unfortunately it does 
    not change the scale of the text operations (yet?). A `scalevalue' of 
    1 is the default. It is allowed to use floating point values.
\item \KeyWord{SetOffset}(x\_offset,y\_offset) \hfill \\
    Adds the offset values to all coordinates at the \TeX{} level. This 
    makes it easier to move figures around.
\item \KeyWord{SetScaledOffset}(x\_offset,y\_offset) \hfill \\
    Adds the offset values to all coordinates at the postscript level. 
    This is done after scaling has been applied. Hence one can work with 
    the scaled coordinates. This can be very handy when drawing curves.
\item \KeyWord{SetWidth}\{widthvalue\} \hfill \\
    Changes the linewidth in all graphics operations. It does 
    not change the linewidth of the text operations. That is a matter of 
    font selection. A `widthvalue' of 
    0.5 is the default. It is allowed to use floating point values.
\item \KeyWord{Text}(x,y)$[$mode$]$\{text\} \hfill \\
    Places a text. The focal point is (x,y). The text is the last 
    parameter. The mode parameter tells how the text should be 
    positioned with respect to the focal point. If this parameter is 
    omitted the center of the text will correspond to the focal point. 
    Other options are: l for having the left side correspond to the 
    focal point, r for having the right side correspond to it, t for 
    having the top at the focal point and b for the bottom. One may 
    combine two letters as in $[$bl$]$, as long as it makes sense.
\item \KeyWord{Vertex}(x,y)\{r\} \hfill \\
    Draws a fat dot at (x,y). The radius of the dot is given by r.
\item \KeyWord{ZigZag}($x_1$,$y_1$)($x_2$,$y_2$)\{amplitude\}\{wiggles\} \hfill \\
    Draws a zigzag line from ($x_1$,$y_1$) to ($x_2$,$y_2$). The width of the 
    zigzagging will be twice the value of `amplitude'. The number of zigzags 
    is given by the last parameter. If twice this parameter is not an integer 
    it will be rounded to an integer value. Whether the first zigzag 
    starts up or down can be influenced with the sign of the amplitude.
\end{itemize}

A note about color. The names of the colors can be found in the local file 
colordvi.sty or colordvi.tex. This file gives also the commands that allow 
the user to change the color of the text.

\section{Examples}

Although the previous section contains all the commands and their proper 
syntax a few examples may be helpful. 
\subsection{Text modes}
The meaning of the mode characters in the text commands can best be 
demonstrated. The statements
\IfColor{\textBlue}{}
\begin{verbatim}
    \begin{center}  \begin{picture}(300,100)(0,0)
    \SetColor{BrickRed}
    \CArc(50,75)(2,0,360)  \Text(50,75)[lt]{left-top}
    \CArc(50,50)(2,0,360)  \Text(50,50)[l]{left-center}
    \CArc(50,25)(2,0,360)  \Text(50,25)[lb]{left-bottom}
    \CArc(150,75)(2,0,360) \Text(150,75)[t]{center-top}
    \CArc(150,50)(2,0,360) \Text(150,50)[]{center-center}
    \CArc(150,25)(2,0,360) \Text(150,25)[b]{center-bottom}
    \CArc(250,75)(2,0,360) \Text(250,75)[rt]{right-top}
    \CArc(250,50)(2,0,360) \Text(250,50)[r]{right-center}
    \CArc(250,25)(2,0,360) \Text(250,25)[rb]{right-bottom}
    \end{picture}   \end{center}
\end{verbatim}
\IfColor{\textBlack}{}
produce 9 texts and for each the focal point is indicated by a little 
circle. It looks like
\begin{center}
\begin{picture}(300,100)(0,0)
\SetColor{BrickRed}
\CArc(50,75)(2,0,360)  \Text(50,75)[lt]{left-top}
\CArc(50,50)(2,0,360)  \Text(50,50)[l]{left-center}
\CArc(50,25)(2,0,360)  \Text(50,25)[lb]{left-bottom}
\CArc(150,75)(2,0,360) \Text(150,75)[t]{center-top}
\CArc(150,50)(2,0,360) \Text(150,50)[]{center-center}
\CArc(150,25)(2,0,360) \Text(150,25)[b]{center-bottom}
\CArc(250,75)(2,0,360) \Text(250,75)[rt]{right-top}
\CArc(250,50)(2,0,360) \Text(250,50)[r]{right-center}
\CArc(250,25)(2,0,360) \Text(250,25)[rb]{right-bottom}
\end{picture}
\end{center}
This illustrates exactly all the combinations of the mode characters and 
what their effects are. The commands $\backslash$Text and 
$\backslash$rText give a tex according to \LaTeX{}. This text is 
insensitive to the scaling commands, and the color of the text should be 
set with the regular color commands given in colordvi.sty. The text in the 
$\backslash$PText command (and the various boxes with text) is a postscript 
text. Such text is sensitive to the scaling commands and in addition the 
color is set with the $\backslash$SetColor command or in the command itself 
(in the case of the boxes). In the case of \LaTeX{} text 
it can of course contain different fonts, 
math mode and all those little things that are usually easier in 
\LaTeX{} than in postscript.

\subsection{The windings of a gluon}

Gluons are traditionally represented by a two dimensional projection of 
a helix. Actually close inspection of some pretty gluons reveals that 
it is usually not quite a helix. Hence the gluons in Axodraw are also 
not quite helices. In addition one may notice that the begin and end 
points deviate slightly from the regular windings. This makes it more in 
agreement with hand drawn gluons. When a gluon is drawn, one needs not 
only its begin and end points but there is an amplitude connected to 
this almost helix, and in addition there are windings. The number of 
windings is the number of curls that the gluon will have. Different 
people may prefer different densities of curls. This can effect the 
appearance considerably:
\IfColor{\textBlue}{}
\begin{verbatim}
    \begin{center}
    \begin{picture}(330,100)(0,0)
    \SetColor{Red}
    \Gluon(25,15)(25,95){5}{4}     \Text(25,7)[]{4 windings}
    \Gluon(95,15)(95,95){5}{5}     \Text(95,7)[]{5 windings}
    \Gluon(165,15)(165,95){5}{6}   \Text(165,7)[]{6 windings}
    \Gluon(235,15)(235,95){5}{7}   \Text(235,7)[]{7 windings}
    \Gluon(305,15)(305,95){5}{8}   \Text(305,7)[]{8 windings}
    \end{picture}
    \end{center}
\end{verbatim}
\IfColor{\textBlack}{}
This code results in:
\begin{center}
\begin{picture}(330,100)(0,0)
\SetColor{Red}
\Gluon(25,15)(25,95){5}{4}    \Text(25,7)[]{4 windings}
\Gluon(95,15)(95,95){5}{5}    \Text(95,7)[]{5 windings}
\Gluon(165,15)(165,95){5}{6}  \Text(165,7)[]{6 windings}
\Gluon(235,15)(235,95){5}{7}  \Text(235,7)[]{7 windings}
\Gluon(305,15)(305,95){5}{8}  \Text(305,7)[]{8 windings}
\end{picture}
\end{center}
The influence of the amplitude is also rather great. The user should 
experiment with it. There is however an aspect to the amplitude that 
should be discussed. For a straight gluon the amplitude can determine on 
which side the curls are. So does the direction of the gluon:
\IfColor{\textBlue}{}
\begin{verbatim}
    \begin{center}
    \begin{picture}(325,100)(0,0)
    \SetColor{Red}
    \Gluon(50,15)(50,95){5}{6}
    \Text(50,7)[]{amp $> 0$}  \Text(40,50)[]{$\uparrow$}
    \Gluon(125,95)(125,15){5}{6}
    \Text(125,7)[]{amp $> 0$} \Text(115,50)[]{$\downarrow$}
    \Gluon(200,15)(200,95){-5}{6}
    \Text(200,7)[]{amp $< 0$} \Text(190,50)[]{$\uparrow$}
    \Gluon(275,95)(275,15){-5}{6}
    \Text(275,7)[]{amp $< 0$} \Text(265,50)[]{$\downarrow$}
    \end{picture}
    \end{center}
\end{verbatim}
\IfColor{\textBlack}{}
The picture gets the following appearance:
\begin{center}
\begin{picture}(325,100)(0,0)
\SetColor{Red}
\Gluon(50,15)(50,95){5}{6}
\Text(50,7)[]{amp $> 0$}  \Text(40,50)[]{$\uparrow$}
\Gluon(125,95)(125,15){5}{6}
\Text(125,7)[]{amp $> 0$} \Text(115,50)[]{$\downarrow$}
\Gluon(200,15)(200,95){-5}{6}
\Text(200,7)[]{amp $< 0$} \Text(190,50)[]{$\uparrow$}
\Gluon(275,95)(275,15){-5}{6}
\Text(275,7)[]{amp $< 0$} \Text(265,50)[]{$\downarrow$}
\end{picture}
\end{center}
For straight gluons one does not need the option of the negative 
amplitude. It is however necessary for gluons on an arc segment. In that 
case the arc is always drawn in an anticlockwise direction. Hence the 
direction is fixed and only the amplitude is left as a tool for 
determining the side with the curls.

\subsection{Scaling}

Sometimes it is much easier to design a figure on a larger scale than it 
is needed in the eventual printing. In that case one can use a scale 
factor, either during the design or in the final result. We use the 
figure in the first section as an example:
\IfColor{\textBlue}{}
\begin{verbatim}
    \vspace{-10pt} \hfill \\
    \SetScale{0.3}
    \begin{picture}(70,30)(0,13)
    \SetColor{Red}
    \GlueArc(120,50)(40,0,180){5}{8}
    \SetColor{Green}
    \GlueArc(120,50)(40,180,360){5}{8}
    \SetColor{Blue}
    \Gluon(20,50)(80,50){5}{4}  \Vertex(80,50){2} 
    \Gluon(160,50)(220,50){5}{4} \Vertex(160,50){2}
    \end{picture} $+$ others
            $ = C_A(\frac{5}{3}+\frac{31}{9}\epsilon)
              + n_F(-\frac{2}{3}-\frac{10}{9}\epsilon)$
    \vspace{10pt}  \hfill \\
\end{verbatim}
\IfColor{\textBlack}{}
We have lowered the figure by 13 points (the (0,13) in the picture 
statement) to make it look nice with respect to the equal sign. The 
result is
    \vspace{-10pt} \hfill \\
    \SetScale{0.3}
    \begin{picture}(70,30)(0,13)
    \SetColor{Red}
    \GlueArc(120,50)(40,0,180){5}{8}
    \SetColor{Green}
    \GlueArc(120,50)(40,180,360){5}{8}
    \SetColor{Blue}
    \Gluon(20,50)(80,50){5}{4}  \Vertex(80,50){2} 
    \Gluon(160,50)(220,50){5}{4} \Vertex(160,50){2}
    \end{picture} $+$ others
            $ = C_A(\frac{5}{3}+\frac{31}{9}\epsilon)
              + n_F(-\frac{2}{3}-\frac{10}{9}\epsilon)$
    \vspace{10pt}  \hfill \\
    \SetScale{1}
This way it is rather straightforward to make whole pictorial equations.
Of course some things are not scale invariant. The appreciation of a 
figure may be somewhat different when the scale is changed. In the above 
case one might consider changing the amplitude of the gluons a little 
bit. Changing this from 5 to 7 and at the same time reducing the number 
of windings from 4 to 3 for the straight gluons and from 8 to 7 for the 
gluons in the arcs gives
    \vspace{-10pt} \hfill \\
    \SetScale{0.3}
    \begin{picture}(70,30)(0,13)
    \SetColor{Red}
    \GlueArc(120,50)(40,0,180){7}{7}
    \SetColor{Green}
    \GlueArc(120,50)(40,180,360){7}{7}
    \SetColor{Blue}
    \Gluon(20,50)(80,50){7}{3}  \Vertex(80,50){2} 
    \Gluon(160,50)(220,50){7}{3} \Vertex(160,50){2}
    \end{picture} $+$ others
            $ = C_A(\frac{5}{3}+\frac{31}{9}\epsilon)
              + n_F(-\frac{2}{3}-\frac{10}{9}\epsilon)$
    \vspace{10pt}  \hfill \\
    \SetScale{1}
At this scale this may please the eye more.

There is one problem with scaling. Currently it is only possible to have 
text scale with the rest of a figure when the text has been printed with 
the PText command. This makes the typesetting more complicated, but the 
scaling of the \TeX{} pixel fonts would give rather poor results anyway.

\subsection{Photons}

When drawing photons one should take care that the number of wiggles is 
selected properly. Very often this number should be an integer plus 
$0.5$. This can be seen in the following example:
\IfColor{\textBlue}{}
\begin{verbatim}
    \begin{center}\begin{picture}(300,56)(0,0)
    \Vertex(180,10){1.5} \Vertex(120,10){1.5}
    \SetColor{Red}
    \ArrowLine(100,10)(200,10)
    \SetColor{Green}
    \LongArrowArc(150,10)(20,60,120)
    \SetColor{Brown}
    \PhotonArc(150,10)(30,0,180){4}{8.5}   % 8.5 wiggles
    \end{picture} \end{center}
\end{verbatim}
\IfColor{\textBlack}{}
This gives the `proper' picture as it would usually drawn by hand:
    \begin{center}\begin{picture}(300,56)(0,0)
    \Vertex(180,10){1.5} \Vertex(120,10){1.5}
    \SetColor{Red}
    \ArrowLine(100,10)(200,10)
    \SetColor{Green}
    \LongArrowArc(150,10)(20,60,120)
    \SetColor{Brown}
    \PhotonArc(150,10)(30,0,180){4}{8.5}   % 8.5 wiggles
    \end{picture} \end{center}
When the number of wiggles is reduced to 8 we obtain:
    \begin{center}\begin{picture}(300,56)(0,0)
    \Vertex(180,10){1.5} \Vertex(120,10){1.5}
    \SetColor{Red}
    \ArrowLine(100,10)(200,10)
    \SetColor{Green}
    \LongArrowArc(150,10)(20,60,120)
    \SetColor{Brown}
    \PhotonArc(150,10)(30,0,180){4}{8}   % 8 wiggles
    \end{picture} \end{center}
This is not as nice. Somehow the symmetry is violated. One should also 
take care that the wiggles start in the proper way. If we make the 
amplitude negative we see that the photons are not `right' either:
    \begin{center}\begin{picture}(300,56)(0,0)
    \Vertex(180,10){1.5} \Vertex(120,10){1.5}
    \SetColor{Red}
    \ArrowLine(100,10)(200,10)
    \SetColor{Green}
    \LongArrowArc(150,10)(20,60,120)
    \SetColor{Brown}
    \PhotonArc(150,10)(30,0,180){-4}{8.5}   % 8.5 wiggles
    \end{picture} \end{center}
Sometimes these things require some experimenting.

\subsection{Flowcharts}

There are several commands for creating boxes with text in them. This can 
be a box with either one line of text or with two lines of text. The rest 
is just a matter of drawing lines and circle segments with arrows. If the 
text is to scale with the picture one needs to use the postscript fonts. 
The result of scaling the {\TeX} fonts is usually rather ugly, because 
these fonts are pixel fonts. Here we present an example. It might describe 
a system for the automatic computation of cross-sections:
\IfColor{\textBlue}{}
\begin{verbatim}
\begin{center} \begin{picture}(320,320)(0,0)
\SetPFont{Helvetica}{10}
\SetScale{0.8}
\SetColor{Magenta}
\ArrowLine(200,40)(200,10)   \ArrowLine(200,100)(200,40)
\ArrowLine(200,150)(200,100) \ArrowLine(100,130)(200,100)
\ArrowLine(85,95)(200,100)   \ArrowLine(260,105)(200,100)
\ArrowLine(250,135)(200,100) \ArrowLine(160,75)(200,100)
\ArrowLine(200,100)(250,70)  \ArrowLine(200,185)(200,150)
\ArrowLine(200,220)(200,185) \ArrowLine(200,250)(200,220)
\ArrowLine(240,263)(200,250) \ArrowLine(240,237)(200,250)
\ArrowLine(200,285)(200,250) \ArrowLine(200,310)(200,285)
\ArrowLine(200,335)(200,310) \ArrowLine(180,360)(200,335)
\ArrowLine(200,385)(180,360) \ArrowLine(50,370)(180,360)
\ArrowArc(200,247.5)(62.5,90,180)
\ArrowArc(200,247.5)(62.5,180,270)
\ArrowLine(210,385)(300,360) \ArrowLine(210,335)(300,360)
\ArrowLine(80,300)(80,130)   \ArrowLine(190,335)(80,300)
\ArrowLine(190,385)(80,300)  \ArrowLine(50,335)(80,300)
\ArrowLine(300,360)(340,340) \ArrowArcn(205,347.5)(37.5,90,270)
\SetColor{Blue}
\BCirc(200,100){10}  \BCirc(200,100){5}
\BCirc(200,40){7.5}  \BCirc(200,250){10}
\BCirc(200,250){5}   \BCirc(200,310){7.5}
\BCirc(180,360){7.5} \BCirc(80,300){7.5}
\BCirc(300,360){7.5}
\IfColor{\CCirc(200,185){7.5}{Blue}{Yellow}
}{\GCirc(200,185){7.5}{0.9}}
\SetColor{Red}
\BText(200,285){Form program} \BText(200,335){Diagrams}
\BText(200,385){Model}        \BText(200,10){events}
\BText(80,95){Axolib}         \BText(350,335){Pictures}
\IfColor{\CText(137.5,247.5){Blue}{Yellow}{instructions}
}{\GText(137.5,247.5){0.9}{instructions}}
\B2Text(260,70){Cross-sections}{Histograms}
\B2Text(140,75){Monte Carlo}{Routine}
\B2Text(275,105){FF}{1 loop integrals}
\IfColor{\C2Text(260,135){Blue}{Yellow}{Spiderlib}{Fortran/C}
}{\G2Text(260,135){0.9}{Spiderlib}{Fortran/C}}
\IfColor{\C2Text(200,150){Blue}{Yellow}{Matrix}{Element}
}{\G2Text(200,150){0.9}{Matrix}{Element}}
\B2Text(80,130){Kinematics}{Configuration}
\IfColor{\C2Text(200,220){Blue}{Yellow}{Output}{Formula}
}{\G2Text(200,220){0.9}{Output}{Formula}}
\IfColor{\C2Text(260,263){Blue}{Yellow}{Spiderlib}{Form part}
}{\G2Text(260,263){0.9}{Spiderlib}{Form part}}
\IfColor{\C2Text(260,237){Blue}{Yellow}{FF support}{library}
}{\G2Text(260,237){0.9}{FF support}{library}}
\B2Text(40,370){Reaction}{selection}
\B2Text(40,340){Specification}{Cuts, etc.}
\SetColor{Orange}
\PText(211,36)(0)[lb]{Event Generator}
\PText(211,181)(0)[lb]{Code Generator}
\PText(162,258)(0)[lb]{FORM}
\PText(211,301)(0)[lb]{Form program construction}
\PText(191,362)(0)[lb]{Diagram}
\PText(191,352)(0)[lb]{Generator}
\PText(311,370)(0)[lb]{Postscript}
\PText(311,360)(0)[lb]{Generator}
\PText(91,292)(0)[lb]{Kinematics}
\PText(91,282)(0)[lb]{Generator}
\end{picture} \end{center}
\end{verbatim}
\IfColor{\textBlack}{}
This gives the chart
\begin{center} \begin{picture}(320,320)(0,0)
\SetPFont{Helvetica}{10}
\SetScale{0.8}
\SetColor{Magenta}
\ArrowLine(200,40)(200,10)   \ArrowLine(200,100)(200,40)
\ArrowLine(200,150)(200,100) \ArrowLine(100,130)(200,100)
\ArrowLine(85,95)(200,100)   \ArrowLine(260,105)(200,100)
\ArrowLine(250,135)(200,100) \ArrowLine(160,75)(200,100)
\ArrowLine(200,100)(250,70)  \ArrowLine(200,185)(200,150)
\ArrowLine(200,220)(200,185) \ArrowLine(200,250)(200,220)
\ArrowLine(240,263)(200,250) \ArrowLine(240,237)(200,250)
\ArrowLine(200,285)(200,250) \ArrowLine(200,310)(200,285)
\ArrowLine(200,335)(200,310) \ArrowLine(180,360)(200,335)
\ArrowLine(200,385)(180,360) \ArrowLine(50,370)(180,360)
\ArrowArc(200,247.5)(62.5,90,180)
\ArrowArc(200,247.5)(62.5,180,270)
\ArrowLine(210,385)(300,360) \ArrowLine(210,335)(300,360)
\ArrowLine(80,300)(80,130)   \ArrowLine(190,335)(80,300)
\ArrowLine(190,385)(80,300)  \ArrowLine(50,335)(80,300)
\ArrowLine(300,360)(340,340) \ArrowArcn(205,347.5)(37.5,90,270)
\SetColor{Blue}
\BCirc(200,100){10}  \BCirc(200,100){5}
\BCirc(200,40){7.5}  \BCirc(200,250){10}
\BCirc(200,250){5}   \BCirc(200,310){7.5}
\BCirc(180,360){7.5} \BCirc(80,300){7.5}
\BCirc(300,360){7.5}
\IfColor{\CCirc(200,185){7.5}{Blue}{Yellow}
}{\GCirc(200,185){7.5}{0.9}}
\SetColor{Red}
\BText(200,285){Form program} \BText(200,335){Diagrams}
\BText(200,385){Model}        \BText(200,10){events}
\BText(80,95){Axolib}         \BText(350,335){Pictures}
\IfColor{\CText(137.5,247.5){Blue}{Yellow}{instructions}
}{\GText(137.5,247.5){0.9}{instructions}}
\B2Text(260,70){Cross-sections}{Histograms}
\B2Text(140,75){Monte Carlo}{Routine}
\B2Text(275,105){FF}{1 loop integrals}
\IfColor{\C2Text(260,135){Blue}{Yellow}{Spiderlib}{Fortran/C}
}{\G2Text(260,135){0.9}{Spiderlib}{Fortran/C}}
\IfColor{\C2Text(200,150){Blue}{Yellow}{Matrix}{Element}
}{\G2Text(200,150){0.9}{Matrix}{Element}}
\B2Text(80,130){Kinematics}{Configuration}
\IfColor{\C2Text(200,220){Blue}{Yellow}{Output}{Formula}
}{\G2Text(200,220){0.9}{Output}{Formula}}
\IfColor{\C2Text(260,263){Blue}{Yellow}{Spiderlib}{Form part}
}{\G2Text(260,263){0.9}{Spiderlib}{Form part}}
\IfColor{\C2Text(260,237){Blue}{Yellow}{FF support}{library}
}{\G2Text(260,237){0.9}{FF support}{library}}
\B2Text(40,370){Reaction}{selection}
\B2Text(40,340){Specification}{Cuts, etc.}
\SetColor{Orange}
\PText(211,36)(0)[lb]{Event Generator}
\PText(211,181)(0)[lb]{Code Generator}
\PText(162,258)(0)[lb]{FORM}
\PText(211,301)(0)[lb]{Form program construction}
\PText(191,362)(0)[lb]{Diagram}
\PText(191,352)(0)[lb]{Generator}
\PText(311,370)(0)[lb]{Postscript}
\PText(311,360)(0)[lb]{Generator}
\PText(91,292)(0)[lb]{Kinematics}
\PText(91,282)(0)[lb]{Generator}
\end{picture} \end{center}

\subsection{Curves and graphs}

Axodraw is equipped with a curve fitting facility that can draw smooth 
curves through a set of coordinates. Coupled to this is a set of 
commands to draw the axes that are typically needed for the use of 
graphs and histograms. An example of a complete picture would be
\IfColor{\textBlue}{}
\begin{verbatim}
\begin{center} \begin{picture}(360,440)(0,0)
\SetOffset(40,30)
\LinAxis(0,0)(300,0)(3,10,5,0,1.5)
\LinAxis(0,400)(300,400)(3,10,-5,0,1.5)
\LogAxis(0,0)(0,400)(4,-5,2,1.5)
\LogAxis(300,0)(300,400)(4,5,2,1.5)
\SetScale{100.} \SetWidth{0.005}
\SetColor{Blue}
\Curve{(.1057001,1.2997)(.1057003,1.5399)
(.1057006,1.6908)(.1057010,1.8019)(.1057030,2.0406)
(.1057060,2.1911)(.1057100,2.3020)(.1057300,2.5403)
(.1057600,2.6904)(.1058000,2.8007)(.1060000,3.0365)
(.1080000,3.4512)(.1100000,3.5600)(.1200000,3.6950)
(.1300000,3.6969)(.1500000,3.6308)(.1800000,3.5024)
(.2200000,3.3413)(.3000000,3.0788)(.5000000,2.6374)
(.8000000,2.2295)(1.0000000,2.0357)(1.3000000
,1.8078)(1.6000000,1.6275)(2.0000000,1.4336)
(2.5000000,1.2398)(3.0000000,1.0815)}
\SetColor{Red}
\DashCurve{(1.7853600,.0111)(1.7853800,.0228)
(1.7854000,.0339)(1.7856000,.1218)(1.7860000,.2324)
(1.7870000,.3821)(1.7900000,.5786)(1.8000000,.8089)
(1.8200000,.9765)(1.8500000,1.0869)(1.9000000
,1.1718)(2.0000000,1.2335)(2.1000000,1.2468)
(2.2000000,1.2413)(2.4000000,1.2064)
(2.7000000,1.1340)(3.0000000,1.0574)}{0.05}
\SetScale{1.}\SetWidth{0.5}
\SetColor{Blue}
\Line(200,360)(270,360)
\Text(195,360)[r]{\large$e^+e^-\rightarrow\mu^+\mu^-$}
\SetColor{Red}
\DashLine(200,330)(270,330){5}
\Text(195,330)[r]{\large$e^+e^-\rightarrow\tau^+\tau^-$}
\SetColor{Black}
\Text(0,-10)[]{0} \Text(100,-10)[]{1}
\Text(200,-10)[]{2} \Text(300,-10)[]{3}
\Text(150,-25)[]{\large Beam energy in GeV}
\Text(-10,70)[]{$1$} \Text(-10,170)[]{$10$}
\Text(-10,270)[]{$10^2$} \Text(-10,370)[]{$10^3$}
\rText(-25,220)[][l]{\Large$\sigma$ in nb}
\ArrowLine(190,270)(160,300)
\ArrowLine(160,240)(190,270)
\ArrowLine(270,300)(240,270)
\ArrowLine(240,270)(270,240)
\Photon(190,270)(240,270){4}{4.5}
\Vertex(190,270){1.5} \Vertex(240,270){1.5}
\end{picture}  \\ {\sl \hskip 10 pt Threshold
effects for $\mu$ and $\tau$} \end{center}
\end{verbatim}
\IfColor{\textBlack}{}
and the resulting picture would be
\begin{center} \begin{picture}(360,440)(0,0)
\SetOffset(40,30)
\LinAxis(0,0)(300,0)(3,10,5,0,1.5)
\LinAxis(0,400)(300,400)(3,10,-5,0,1.5)
\LogAxis(0,0)(0,400)(4,-5,2,1.5)
\LogAxis(300,0)(300,400)(4,5,2,1.5)
\SetScale{100.} \SetWidth{0.005}
\SetColor{Blue}
\Curve{(.1057001,1.2997)(.1057003,1.5399)
(.1057006,1.6908)(.1057010,1.8019)(.1057030,2.0406)
(.1057060,2.1911)(.1057100,2.3020)(.1057300,2.5403)
(.1057600,2.6904)(.1058000,2.8007)(.1060000,3.0365)
(.1080000,3.4512)(.1100000,3.5600)(.1200000,3.6950)
(.1300000,3.6969)(.1500000,3.6308)(.1800000,3.5024)
(.2200000,3.3413)(.3000000,3.0788)(.5000000,2.6374)
(.8000000,2.2295)(1.0000000,2.0357)(1.3000000
,1.8078)(1.6000000,1.6275)(2.0000000,1.4336)
(2.5000000,1.2398)(3.0000000,1.0815)}
\SetColor{Red}
\DashCurve{(1.7853600,.0111)(1.7853800,.0228)
(1.7854000,.0339)(1.7856000,.1218)(1.7860000,.2324)
(1.7870000,.3821)(1.7900000,.5786)(1.8000000,.8089)
(1.8200000,.9765)(1.8500000,1.0869)(1.9000000
,1.1718)(2.0000000,1.2335)(2.1000000,1.2468)
(2.2000000,1.2413)(2.4000000,1.2064)
(2.7000000,1.1340)(3.0000000,1.0574)}{0.05}
\SetScale{1.}\SetWidth{0.5}
\SetColor{Blue}
\Line(200,360)(270,360)
\Text(195,360)[r]{\large$e^+e^-\rightarrow\mu^+\mu^-$}
\SetColor{Red}
\DashLine(200,330)(270,330){5}
\Text(195,330)[r]{\large$e^+e^-\rightarrow\tau^+\tau^-$}
\SetColor{Black}
\Text(0,-10)[]{0} \Text(100,-10)[]{1}
\Text(200,-10)[]{2} \Text(300,-10)[]{3}
\Text(150,-25)[]{\large Beam energy in GeV}
\Text(-10,70)[]{$1$} \Text(-10,170)[]{$10$}
\Text(-10,270)[]{$10^2$} \Text(-10,370)[]{$10^3$}
\rText(-25,220)[][l]{\Large$\sigma$ in nb}
\ArrowLine(190,270)(160,300)
\ArrowLine(160,240)(190,270)
\ArrowLine(270,300)(240,270)
\ArrowLine(240,270)(270,240)
\Photon(190,270)(240,270){4}{4.5}
\Vertex(190,270){1.5} \Vertex(240,270){1.5}
\end{picture}  \\ {\sl \hskip 10 pt Threshold
effects for $\mu$ and $\tau$} \end{center}
Of course one can scale these pictures further, but because the scale 
factor has been used to enter the data points these should then be 
adapted too. Note that when the scale is blown up by a factor 100, the 
linewidth has to be scaled down or disasters will take place.

Finally a playful example:
\IfColor{\textBlue}{}
\begin{verbatim}
    \begin{center}\begin{picture}(300,56)(0,0)
    \SetColor{Blue}
    \Line(100,25)(150,25)
    \SetColor{Green}
    \Gluon(150,25)(200,25){3}{6}
    \SetColor{Red}
    \Photon(150,35)(200,45){3}{6}
    \SetColor{Mahogany}
    \ZigZag(150,15)(200,5){3}{6}
    \IfColor{\COval(150,25)(20,10)(0){Black}{Yellow}
         }{\GOval(150,25)(20,10)(0){0.5}}
    \end{picture} \end{center}
\end{verbatim}
\IfColor{\textBlack{}}{}which results in
    \begin{center}\begin{picture}(300,56)(0,0)
    \SetColor{Blue}
    \Line(100,25)(150,25)
    \SetColor{Green}
    \Gluon(150,25)(200,25){3}{6}
    \SetColor{Red}
    \Photon(150,35)(200,45){3}{6}
    \SetColor{Mahogany}
    \ZigZag(150,15)(200,5){3}{6}
    \IfColor{\COval(150,25)(20,10)(0){Black}{Yellow}
         }{\GOval(150,25)(20,10)(0){0.5}}
    \end{picture} \end{center}

Acknowledgement: The author wishes to thank G.J.van Oldenborgh for help 
with some of the \TeX{} macros.
 
Axodraw can be obtained from the authors homepage:
\IfColor{\textOliveGreen}{}{http://norma.nikhef.nl/$\sim$t68/axodraw}~
{\IfColor{\textBlack}{}$\!\!\!\!$.}
Alternatively it is available by means of anonymous ftp from ftp.nikhef.nl. 
There it is located in the directory pub/form/axodraw. Commentary and 
suggestions should be sent to the author at t68\verb:@:nikhef.nl.
\end{document}
